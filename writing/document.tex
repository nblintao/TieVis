\documentclass{article}
\title{Report 2015/7/23}
\author{Tao LIN}

\begin{document}

\maketitle

\section{Related Work}

\subsection{Large Graph Layout}

Many graph layout algorithms have been created over the past decades. 
GRIP\cite{gajer2000fast,gajer2001grip} is a hierarchical force-directed method for drawing large graphs.
OpenOrd\cite{martin2011openord} has both serial and parallel implementation of large graph layout.
Tikhonova et al.\cite{tikhonova2008scalable} presented a scalable parallel graph layout algorithm with good performance, scalability and the visual quality.

Is essential to make comparison and decide the most suitable algorithms for the specific scenario. Gibson et al.\cite{gibson2013survey} discussed a range of two-dimensional graph layout techniques for information visualization. Hachul et al.\cite{hachul2006experimental} compared several large graph drawing algorithms by the differences in drawing qualities and running times.
 


\subsection{Level-of-Detail and Adaptive Techniques for Large Graph}

Even it is available to layout large graphs in real time, it is difficult to obtain useful information from the visualization, since the size of display devices and users' cognition abilities are limited. Hence, the simplification should be considered.

Level-of-detail techniques are able to simplify the large graphs by applying graph clustering and summarization. ASK-GraphView\cite{abello2006ask} is a graph visualization system that allows clustering and interactive
navigation of large graphs with interactive rates by the scalable architecture and sophisticated clustering algorithms.
HiMap\cite{shi2009himap} is another system that visualizes by clustered graph via hierarchical grouping and summarization. Moreover, employed a novel adaptive data loading technique considering the visual density of the graph view.

Also, the visual design should be improved considering the levels of detail.
Balzer et al.\cite{balzer2007level} presents level-of-detail techniques for visualizing large and complex clustered graph layouts with implicit surfaces.
Zinsmaier et al.\cite{zinsmaier2012interactive} proposes a combination of edge cumulation with density-based node aggregation for large graphs without precomputed hierarchies.

%Gibson et al.\cite{gibson2011node} showed an optimization for small-world network layout by using attributes to layout the graph in high-dimensional data space.

\subsection{Parallel Computation in Information Visualization}

Parallel computations are often applied in large data analysis and visualization, but almost all of them are applied on scientific data. However, with the increasing size of data in cyberspace, information visualization should also consider parallel computations since traditional ways are often inadequate to visualize in real time.

Here are some successful researches on the parallel computation in information visualization.
Liu et al.\cite{liu2013immens} put up a combination of multivariate data tiles and parallel query processing for real-time interacive querying large data sets.
Shen et al.\cite{shen2012visual} introduced a two-tier visual analytics system, which incorporates a Hadoop-based parallel data processing platform to handle queries of massive web session data.

\subsection{Web-based Information Visualization}

Web-based visualization technologies make more people capable of performing complex analysis tasks conveniently. Many of them employed different system architectures, which could be classified in three categories\cite{bostock2009protovis}: thin client, fat server and hybrid approaches.

Eick et al.\cite{eick2007thin} developed a thin client visualization framework that provides a rich user experience that is completely browser based.
To develop the performance of client visualization, Johnson et al.\cite{johnson2008scalability} contrasted several web-native information visualization methods (SVG, HTML5's Canvas, native HTML) at different data scales.
 
 
For fat server technologies, most computation occurs on the server.
Poliakov et al.\cite{poliakov2005server} described a client-server approach to enable researchers to visualize, manipulate, and analyze large brain imaging datasets over the Internet while all computationally intensive tasks are done by a graphics server.
While Lamberti et al.\cite{lamberti2007streaming} introduced a streaming-based solution for
remote visualization of 3D graphics on mobile devices.

Hybrid approaches are more adaptive to different applications.
Bender et al.\cite{bender2000functional} presented a framework which allows the easy and efficient implementation of general Web-based visualization systems which can either follow the fat server, the fat client, or a hybrid approach.
Wood et al.\cite{wood2008web} separated middleware from the visualization components. 
Pike et al.\cite{pike2009scalable} introduced the scalable reasoning system, which provides web-based and mobile interfaces for visual analysis. This system scales well to large
numbers of potentially distributed users and across the range of interfaces in use within an enterprise.

%A web services architecture for visualization

\bibliographystyle{ieeetr}
\bibliography{references}
\end{document}

